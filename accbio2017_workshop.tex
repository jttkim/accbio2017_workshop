\documentclass[a4paper,fleqn]{article}

\usepackage[english]{babel}
\usepackage[T1]{fontenc}
\usepackage{mathptmx}



\begin{document}

\title{Workshop at Accelerating Biology 2017}
\author{Sunitha Manjari, Jan T Kim, \ldots}
\date{}
\maketitle


\section{Theoretical Foundations}

\subsection{Recap: NGS Techniques}

We will briefly review the basics of NGS, focusing on ``PHRED''
qualities and other concepts important for NGS bioinformatics. NGS
biochemistry may be touched upon but participants are expected to have
the necessary biochemistry background to be able to get themselves up
to speed.


\subsection{NGS Applications}

\subsubsection{Mapping}

This covers basics and limitations (such as limitations to sensitivity
resulting from ``pigeonhole principle''), but will be light on
algorithmics (e.g.\ no details on Burrows-Wheeler etc.).


\subsection{Variant Calling}


\subsection{RNA-Seq}


\section{Technical Introduction}

\subsection{Overview of Files and File Formats}

Brief intro / recap on babylonian sequence file format situation
(using FASTA and GenBank, no more), intro of FASTQ, including
discussion of pitfalls and caveats (single / multiline assumption,
ambiguity of \texttt{@} as record start or quality code etc.).


\subsection{Introduction of the CLI}

Drawing on SWC materials.

Should perhaps include first practical exercise so we get an idea of
participant's laptops and experience / abilities?



\section{Hands-On and Demos}

tbd





\end{document}


%%% Local Variables: 
%%% mode: latex
%%% TeX-master: t
%%% End: 
